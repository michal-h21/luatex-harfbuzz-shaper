\documentclass{article}
\usepackage{harfbuzz}
\begin{document}
\font\amiri={Amiri} at 12pt
\amiri
\startharfbuzz

%\SetFontOption{script}{arab}
\SetFontOption{direction}{RTL}

\SetFontOption{language}{ARA}
\luatextextdir TRT
\luatexpardir TRT

الخط الأميري 

الخط الأميري خط نسخي موجه لطباعة الكتب و النصوص الطويلة.

الخط الأميري هو إحياء و محاكاة للخط الطباعي الجميل الذي تميزت به مطبعة بولاق منذ أوائل القرن العشرين، و التي عرفت أيضا بالمطبعة الأميرية، و من هنا أخذ الخط اسمه.

يتميز خط المطابع الأميرية بجماليته و مراعاته لفن الخط العربي، بأسلوب نسخي جميل، و في ذات الوقت يراعى متطلبات الطباعة و القيود التي تفرضها، من غير إفراط في جانب على حساب الآخر. و لهذا يتميز بمناسبته للصف الطباعي عموما، و لصف الكتب خصوصا. و قد استُخدِم هذا الخط في طباعة المصحف الأميري، و هو من المصاحف القليلة التي طبعت بالصف المعدني و لم يخطها خطاط بيده، و هذا يجعله مادة خصبة لبناء خط حاسوبي مناسب لصف النصوص القرآنية.

\end{document}
