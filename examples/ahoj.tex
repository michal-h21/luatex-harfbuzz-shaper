\documentclass{article}
\usepackage[czech,greek,russian,english]{babel}
\usepackage{luacode}
\usepackage{luatexbase}
\usepackage{microtype}
% to see what does hyphenation
\textwidth=20em
\usepackage{harfbuzz}
\usepackage{url}
% polyglossia doesn't work. it calls fontspec
%\usepackage{polyglossia}
% \begin{luacode*}
% fl = require "hf_fontload"
% \end{luacode*}
\font\ahoj= {TeX Gyre Schola}  at 12pt

\begin{document}
\parindent=0pt
\parskip=1em
%  \selectlanguage{czech}
%  \ahoj Nazdar světe grafika
%  \begin{luacode*}
%    local options = fontloader.options
%    print(font.fonts[font.current()].backmap)
%    fl.write_nodes(
%      fl.make_nodes("grafika řeřicha", {font=font.current(), lang=16}, fl.options)
%    )
%  \end{luacode*}
%  
%  další text

% we don't have support for Devanagari with Babel. Polyglossia doesn't work
% it loads Fontspec and it causess mass destruction. Or at least fatal error.
\selectlanguage{english}
\frenchspacing % two spaces after dot are horrible

\font\siddhanta={Siddhanta:script=deva;language=HIN} at 12pt
\bigskip

\siddhanta 

Normal devanagari shaping: 

\stopharfbuzz

॥ धर्मो रक्षति रक्षितः ॥

\startharfbuzz

and with harfbuzz: 


॥ धर्मो रक्षति रक्षितः ॥
%  \begin{luacode*}
%    local options = fl.options
%  fl.write_nodes(
%    fl.make_nodes("॥ धर्मो रक्षति रक्षितः ॥",{font = font.current(), lang=0}, options)
%   )
%  \end{luacode*}

\ahoj

% \begin{luacode*}
% luatexbase.add_to_callback("pre_linebreak_filter",fl.process_nodes, "xxx")
% \end{luacode*}

Now some examples with the node callback. All following texts were processed by
Harfbuzz from TeX\ nodes: grafika graf\/ika žluva shelfful shelf\/ful


\selectlanguage{czech}
Nějaký text v češtině, zajímá mě, jak bude fungovat dělení slov, především ve
slovech s diakritikou. Příliš žluťoučký kůň úpěl ďábelské ódy. 

\selectlanguage{english}

\font\amiri={Scheherazade} at 12pt
\amiri
%  \begin{luacode*}
%  --fl.options.direction = "RTL"
%  fl.options.script = "arab"
%  \end{luacode*}
Now some more hardcore example:

\SetFontOption{script}{arab}
\SetFontOption{language}{ARA}
%\SetFontOption{direction}{RTL}
\bgroup
\textdir TRT
\pardir TRT

 الخط الأميري {\textdir TLT hello}

\egroup

\ahoj
\SetFontOption{script}{latn}
\SetFontOption{direction}{LTR}

This is  Arabic text with ``Scheherazade''font. (as we can see -- TeX ligatures
don't work. These must be processed separately from Lua callbacks). 
It works with this font, but fails with ``Amiri''. 

\siddhanta
\def\UrlFont{\siddhanta}
So we can rather try some Devanagari, the text is copied from Wikipedie\footnote{\url{https://hi.wikipedia.org/wiki/चेक_गणराज्य}}

%  \begin{luacode*}
%  fl.options.script = "dflt"
%  \end{luacode*}

चेक गणराज्य मध्य यूरोप में स्थित है। यह सभी ओर से ज़मीन से घिरा हुआ है (अर्थात इसकी किसी सीमा पर समुद्र या महासागर नहीं है)। इसकी सीमाएँ पोलैंड, जर्मनी, ऑस्ट्रिया और स्लोवाकिया से मिलती हैं। इसके मुख्य तीन भाग हैं बोहीमिया, मोराविया और साइलीसिया। राष्ट्र का कुल क्षेत्रफल 30,450 वर्ग मील है, जिसमें से 20,367 वर्ग मील बोहीमिया में हैं। देश की राजधानी प्राग मध्य बोहीमिया में स्थित है।

\font\gentium={Gentium} at 12 pt
% this tries to load bad font, error occurs
\selectlanguage{greek}
\gentium
%  \begin{luacode*}
%  fl.options.script = "dflt"
%  \end{luacode*}
Επαγωγή στο 
Βάση επαγωγής: Για 
Επαγωγική υπόθεση: Υποθέτουμε ότι ισχύει για 
Επαγωγικό βήμα: Θα δείξουμε ότ
Επαγωγή στο
Βάση επαγωγής: Για 

\selectlanguage{russian}
\gentium
%  \begin{luacode*}
%  fl.options.script = "cyrl"
%  \end{luacode*}

\SetFontOption{script}{cyrl}
а также большинство неславянских языков народов СССР, некоторые из которых
ранее имели другие системы письменности (на латинской, арабской или иной
основе) и были переведены на кириллицу в конце 1930-х годов. Подробнее см.
список языков с алфавитами на основе кириллицы.  




\end{document}
