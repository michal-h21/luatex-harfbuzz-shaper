\documentclass{article}
\usepackage[czech,greek,russian,english]{babel}
\usepackage{luacode}
\usepackage{luatexbase}
\usepackage{microtype}
% to see what does hyphenation
\textwidth=20em
\usepackage{harfbuzz}
% polyglossia doesn't work. it calls fontspec
%\usepackage{polyglossia}
% \begin{luacode*}
% fl = require "hf_fontload"
% \end{luacode*}
\font\ahoj= {TeX Gyre Schola}  at 12pt

\begin{document}
\parindent=0pt
\parskip=1em
%  \selectlanguage{czech}
%  \ahoj Nazdar světe grafika
%  \begin{luacode*}
%    local options = fontloader.options
%    print(font.fonts[font.current()].backmap)
%    fl.write_nodes(
%      fl.make_nodes("grafika řeřicha", {font=font.current(), lang=16}, fl.options)
%    )
%  \end{luacode*}
%  
%  další text

% we don't have support for Devanagari with Babel. Polyglossia doesn't work
% it loads Fontspec and it causess mass destruction. Or at least fatal error.
\selectlanguage{english}
\frenchspacing % two spaces after dot are horrible

\font\siddhanta={Siddhanta} at 12pt
\bigskip

\siddhanta 

Normal devanagari shaping: 

॥ धर्मो रक्षति रक्षितः ॥

\startharfbuzz

and with harfbuzz: 


॥ धर्मो रक्षति रक्षितः ॥
%  \begin{luacode*}
%    local options = fl.options
%  fl.write_nodes(
%    fl.make_nodes("॥ धर्मो रक्षति रक्षितः ॥",{font = font.current(), lang=0}, options)
%   )
%  \end{luacode*}

\ahoj

% \begin{luacode*}
% luatexbase.add_to_callback("pre_linebreak_filter",fl.process_nodes, "xxx")
% \end{luacode*}

Now some examples with the node callback. All following texts were processed by
Harfbuzz from TeX\ nodes: grafika graf\/ika žluva shelfful shelf\/ful


\selectlanguage{czech}
Nějaký text v češtině, zajímá mě, jak bude fungovat dělení slov, především ve
slovech s diakritikou. Příliš žluťoučký kůň úpěl ďábelské ódy. 

\selectlanguage{english}

\font\amiri={Scheherazade} at 12pt
\amiri
%  \begin{luacode*}
%  --fl.options.direction = "RTL"
%  fl.options.script = "arab"
%  \end{luacode*}
Now some more hardcore example:

\SetFontOption{script}{arab}
\SetFontOption{language}{ARA}
%\SetFontOption{direction}{RTL}
\bgroup
\luatextextdir TRT
\luatexpardir TRT

 الخط الأميري {\luatextextdir TLT hello}

\egroup

\ahoj
\SetFontOption{script}{latn}
\SetFontOption{direction}{LTR}

This should be ``Amiri Font'' in Arabic. (And as we can see -- TeX ligatures
doesn't work either. These must be processed from Lua). But it doesn't work, as
BIDI obviously isn't easy. Also, maybe we should process whole paragraph at
once, not just words. But that would be difficult because we would loose non
glyph nodes. This clearly needs help from a expert.

\siddhanta
So we can rather try some Devanagari, the text is copied from a comment on http://svayambhava.\-blogspot.com\-/p\--/\-sidd\-hanta-\-deva\-na\-ga\-ri\-unicode-\-open\-type.html

%  \begin{luacode*}
%  fl.options.script = "dflt"
%  \end{luacode*}

महोदय,
मैं भी काशी के संपूर्णानन्द संस्कृत विश्वविद्यालय से हूँ। आप भी काशी में रहे हैं।
निवेदन है कि वैदिका फॉन्ट बहुत उपयोगी है। वाक्यों में (अ) का प्रयोग कम और (ा) मात्रा का प्रयोग बहुत होता है और (ा) को टाइप करने के लिए बार बार शिफ्ट A दबाना पड़ता है। यदि (अ) को शिफ्ट A के साथ तथा (ा) को बिना शिफ्ट a के साथ रखा जाय तो बहुत सुविधा होगी।
धन्यवाद।

आपका
आचार्य रामप्यारे सिंह शास्त्री

\font\gentium={Gentium} at 12 pt
% this tries to load bad font, error occurs
\selectlanguage{greek}
\gentium
%  \begin{luacode*}
%  fl.options.script = "dflt"
%  \end{luacode*}
Επαγωγή στο 
Βάση επαγωγής: Για 
Επαγωγική υπόθεση: Υποθέτουμε ότι ισχύει για 
Επαγωγικό βήμα: Θα δείξουμε ότ
Επαγωγή στο
Βάση επαγωγής: Για 

\selectlanguage{russian}
\gentium
%  \begin{luacode*}
%  fl.options.script = "cyrl"
%  \end{luacode*}

\SetFontOption{script}{cyrl}
а также большинство неславянских языков народов СССР, некоторые из которых
ранее имели другие системы письменности (на латинской, арабской или иной
основе) и были переведены на кириллицу в конце 1930-х годов. Подробнее см.
список языков с алфавитами на основе кириллицы.  




\end{document}
